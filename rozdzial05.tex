\chapter{Dokumentacja techniczna}% 
Z racji tego, że platforma ma obsługiwać zarówno wysokopoziomowe funkcje jak i niskopoziomowe jak GPIO w urządzeniu, to projekt podzielono na 2 części - część serwerową i stronę internetową. W tym rozdziale zostały opisane tylko istotne  części projektu, które rozwiązują istotne problemy lub prezentują organizację kodu. Cały projekt został napisany według zasad opisanych w książce "Clean Code" \cite{cleancode} gdzie jedną z ważniejszych zasad jest brak komentowania kodu. Kod sam siebie powinien opisywać za pomocą swojego nazewnictwa i struktury. 
\section{Aplikacja internetowa}
Przy pomocy strony użytkownik monitoruje i steruje wszystkimi podłączonymi urządzeniami w budynku. Może także zobaczyć informacje pochodzące z czujników w domu np. pomiar temperatury w kuchni. \par
Projekt został napisany w języku javascript ES6+, więc podczas rozwijania aplikacji internetowej posiłkowano się transpilerem. Konwertuje on kod na starsze wersje javascript, które są wspierane przez przeglądarki. Z tego powodu gdy tworzona jest  wersja produkcyjna należy zbudować stronę, a następnie zaimportować ją do serwera. Ten aspekt zostanie dalej opisywany w części serwerowej.
\par Istotnym aspektem aplikacji jest komunikacja z serwerem. Jest ona nawiązywana dwuetapowo.
Serwer ma wyprowadzony endpoint, z którym łączy się strona w celu otrzymania portu, na którym działa websocket
(listing \ref{lst:websocket} \ref{lst:websocketGet}). W trakcie kiedy jeszcze oczekuje się na otrzymanie odpowiedzi od serwera, dodaje się do kolejki akcje, które powinny być wysłane przez websocket (listing \ref{lst:websocket} \ref{lst:websocketQueue}). Po otrzymaniu portu, łączy się websocket do określonego portu (listing \ref{lst:websocket} \ref{lst:websocketCreate}). 
\par Rozwiązanie to sprawia że w przypadku decyzji zmiany portów, ustawia się je tylko na serwerze. Nie ma potrzeby edycji i budowania na nowo aplikacji webowej. Klasa obsługująca websocket została pokazana w listingu \ref{lst:websocket}
\newpage
\begin{lstlisting}[columns=fullflexible,caption={websocket.ts}\label{lst:websocket},language=Java]
import axios from 'axios'
class ReconnectingWebSocket {
    private connection: null | WebSocket = null;
    private onmessageHandler: any;
    private sendQueue: any = [];
    constructor() {
        axios.get(`http://${window.location.hostname}:8080/websocket_port`)
        .then(result => { /*!\annotation{lst:websocketGet}!*/
            this.connection = new WebSocket(
            'ws:' + window.location.hostname + `:${result.data.port}`) /*!\annotation{lst:websocketCreate}!*/
            this.connection.onmessage = this.onmessageHandler;
            this.connection.onerror = (err: any) => console.error(err)
            this.connection.onopen = (resultws: any) => {
                this.sendQueue.forEach(
                    (data: any) => this.connection && this.connection.send(data));
            }
        }).catch((err) => {
            console.error(err)
        })
    };
    set onmessage(handler: any) {
        if (this.connection) {
            this.connection.onmessage = handler;
        } else {
            this.onmessageHandler = handler;
        }
    }
    public send(data: any) {
        if (this.connection !== null && this.connection.readyState === WebSocket.OPEN) {
            this.connection.send(data)
        } else {
            this.sendQueue.push(data); /*!\annotation{lst:websocketQueue}!*/
        }
    }
}
const ws = new ReconnectingWebSocket();
export default ws
\end{lstlisting} \newpage
\par Cała aplikacja została napisana według wzorców opisywanych na głównej stronie React'a  \cite{React}. Oprócz opisywanego kodu w listingu \ref{lst:websocket} nie odchodzi ona od standardów pracy w tym środowisku.\\
Aplikacja posiada tylko dwie strony, które są potrzebne do minimalnego funkcjonowania czyli Dashboard i Settings.
\subsection{Dashboard}
Na rys \ref{fig:dashboard} ukazano wszystkie zapisane urządzenia w postaci kafelek. Wyświetlane są tylko najważniejsze informacje czyli typ urządzenia(Przycisk, czujnik itp.), nazwa i stan. W przypadku przycisku istnieje także możliwość klikania. 
\begin{figure}[h]
  \includegraphics[width=\linewidth]{dashboard.png}
  \caption{Dashboard}
  \label{fig:dashboard}
\end{figure}
\newpage

\subsection{Settings}
Strona służy do konfiguracji naszego systemu. Istnieje tutaj możliwość dodania nowych urządzeń, edycji już istniejących lub wglądu do szczegółów takich jak logi urządzeń.
\begin{figure}[h!]
  \includegraphics[width=\linewidth]{settings.png}
  \caption{Settings}
  \label{fig:settings}
\end{figure}
\newpage
\section{Serwer}
Serwer jest pośrednikiem między poleceniami użytkownika, a urządzeniami. Jednocześnie zapisuje aktualną konfigurację w bazie danych i trzyma logi. Jest on podzielony na komponenty:
\begin{itemize}
    \item http,
    \item baza danych,
    \item API do urządzeń.
\end{itemize}
Został on napisany w języku javascript. Korzysta on z transpilera babel-node w celu pracy w wersjach es6+. Serwer posiada dodatkowe testy, które sprawdzają działanie niektórych komponentów.
\subsection{HTTP}
Serwer korzysta z obszernego framework'a Express \cite{express} przeznaczonego dla języka javascript, który pozwala zaimportować cały katalog zawierający stronę (listing \ref{lst:http} \ref{lst:connectPublicCat}). Zbudowana strona jest przechowywana w katalogu \textit{public}.\\
Oprócz tego obsługiwane jest wspominane już zapytanie o port, na którym działa websocket (listing \ref{lst:websocket} \ref{lst:getWSPort}).
Ten krótki kod w \ref{lst:http} pokazuje całą strukturę http w projekcie.
\begin{lstlisting}[columns=fullflexible,caption={http.js}\label{lst:http},language=Java]
export const listen = config => {
    app.get('/websocket_port', (req, res) => {
        res.json({port: config.WS_PORT}) /*!\annotation{lst:getWSPort}!*/
    })
    app.use(express.static(config.HTMLPath)); /*!\annotation{lst:connectPublicCat}!*/
    app.get('/', (req, res) => {
        res.sendFile(`${config.HTMLPath}/index.html`);
    });
    app.listen(config.HTTP_PORT, () => console.log(`http listening on port ${config.HTTP_PORT}`))
}
\end{lstlisting}
\subsection{Baza Danych}
Jako, że w założeniach projektu zostało postanowione, że powinien on działać na wielu sprzętach, to użyta została relacyjna baza danych na silniku SQLite \cite{sqlite}. Sama baza nie jest rozbudowana i zawiera tylko dwie tabele czyli logi i urządzenia. Czas w logach jest typu int, mimo że SQLite posiada typ danej timestamp. Powodem tej decyzji jest łatwiejsza integracja ze środowiskiem javascript \cite{sqlitejs}.
\begin{figure}[h!]
  \includegraphics[width=\linewidth]{db.png}
  \caption{schemat bazy}
  \label{fig:db}
\end{figure}

Obsługa bazy danych ma przewidziany osobny moduł w serwerze, który upraszcza wiele procesów. W listingu \ref{lst:updateDeviceSQL} została zaprezentowana funkcja do aktualizacji urządzeń w bazie. Posiada ona tylko jeden argument, którym jest urządzenie do zaktualizowania (listing \ref{lst:updateDeviceSQL} \ref{lst:updateDeviceParameter}) Dzięki temu funkcja może być łatwo re-używalna w przypadku integracji nowych urządzeń i pisania do nich komponentów. Funkcja ta buduje komendę SQL (listing \ref{lst:updateDeviceSQL} \ref{lst:updateDeviceBuildCommand}), a następnie dodaje nowy log (listing \ref{lst:updateDeviceSQL} \ref{lst:updateDeviceAddNewLog})  w przypadku gdy wartość urządzenia uległa zmianie (na przykład zmieniła się temperatura).
\newpage
\begin{lstlisting}[columns=fullflexible,caption={edycja urządzenia}\label{lst:updateDeviceSQL},language=Java]
export const updateDevice = (device) => { /*!\annotation{lst:updateDeviceParameter}!*/
	return new Promise(async (resolve, reject) => {
		let command = `UPDATE ${tables.DEVICES} SET`;
		Object.keys(device).forEach(
		    (key)=>key!=='id' && (command+=` ${key}="${device[key]}",`)); /*!\annotation{lst:updateDeviceBuildCommand}!*/
		command = command.slice(0, -1);
		command += ` WHERE id="${device.id}"`;
		const shouldPushLog = await checkIfValueChanged(device);
		if (shouldPushLog) {
			addNewLog(device); /*!\annotation{lst:updateDeviceAddNewLog}!*/
		}

		db.run(command, (err, result) => {
			if (err) {
				console.error(err);
				reject();
			}
			resolve();
		});
	});
};
\end{lstlisting}
Charakter działania wygląda podobnie przy reszcie funkcji z tego modułu, dzięki czemu użytkownik nie potrzebuje wiedzy o strukturze bazy danych w celu korzystania z niej. 

\subsection{MQTT}
Serwer posiada obsługę MQTT w celu komunikacji z urządzeniami automatyki. W ramach projektu zintegrowany został sterownik przekaźnikowy Shelly1, który jest sterowany za pomocą tego protokołu. Moduł ten został napisany w osobnym pliku, a program \textit{mossquitto} pełni funkcję brokera.
\par \ref{lst:initMQTT} pokazuje jedyny spójny element dla każdego modułu wykorzystujący MQTT. W rozdziale 6 została opisana implementacja API w projekcie.

\begin{lstlisting}[columns=fullflexible,caption={inicializacja modułu mqtt}\label{lst:initMQTT},language=Java]
try {
	exec('mosquitto -p 27007', (error, stdout, stderr) => {
		if (error) {
			console.error(error);
		}
	});
} catch (err) { console.log(err) }
\end{lstlisting}
