\chapter{Dokumentacja techniczna}% 
\section{Aplikacja internetowa}
Przy pomocy strony użytkownik monitoruje i steruje wszystkimi podłączonymi urządzeniami w budynku. Może także zobaczyć informacje pochodzące z czujników w domu np. pomiar temperatury w kuchni. \\
Aplikacja posiada 2 strony.
\subsection{Dashboard}
Widzimy tutaj wszystkie zapisane urządzenia w postaci kafelek. Wyświetlane są tylko najważniejsze informacje czyli typ urządzenia(Przycisk, czujnik itp.), nazwa i stan. W przypadku przycisku mamy także możliwość klikania. 
\subsection{Settings}
Strona służy do konfiguracji naszego systemu. Mamy tu możliwość dodania nowych urządzeń, edycji już istniejących lub wglądu do szczegółów takich jak logi urządzeń.
\section{Serwer}
Jest on pośrednikiem między poleceniami użytkownika, a urządzeniami. Jednocześnie zapisuje aktualną konfigurację w bazie danych i trzyma logi. Jest on podzielony na komponenty:
\begin{itemize}
    \item http
    \item baza danych
    \item API do urządzeń
\end{itemize}
