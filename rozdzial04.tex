\chapter{Analiza technologiczna}
Celem projektu było to aby jak największe grono odbiorców było w stanie tworzyć dodatkowe moduły do sterowania. W związku z tym wykorzystano najpopularniejsze technologie. Aby wywołać taki skutek należałoby również zadbać o jakość kodu i trzymanie się ustalonych zasad. Postępowanie takie ma długoterminowe korzyści w postaci mniejszej ilości błędów, łatwości ich naprawiania jak i szybszego wdrażania nowych osób w dany projekt. 
\par Jak już zostało wspomniane w rozdziale 2, w trakcie pracy wzorowano się na istniejących platformach. Głównym elementem był UI i dominacja aplikacji webowych. W inny sposób została zaprojektowana na przykład komunikacja, która u innych jest oparta o REST API, a w projekcje zdecydowano się na wykorzystanie websocket. Nie oznacza to jednak, że całkiem zrezygnowano z zapytań http, ponieważ na przykład konfiguracja websocket została poprzedzona zapytaniem do serwera o port, na którym ma działać.
\section{Zastosowane technologie}
Aktualnie jednym z najpopularniejszych języków jest JavaScript w nowych odsłonach ES6+. Jest to język skryptowy więc na różnych systemach i architekturach powinien działać tak samo. Największą wadą tego języka jest dynamiczne typowanie przez co kod potrafi być mało precyzyjny. W tym celu zdecydowano się na nakładkę Typescript, która wprowadza interface'y znane m.in z języka Java  oraz potrafi wymusić silne typowanie. Dodatkowo strona internetowa powinna być napisana w jednym z popularniejszych frameworków by oszczędzić czas i usprawnić prace nad projektem. Na rynku obecnie dominują trzy blioteki: React, Angular i Vue. W projekcie zdecydowano się na Reacta ze względu na duże wsparcie społeczności oraz składni JSX. \\
Jednym z ważniejszych aspektów platformy jest responsywność, dlatego zamiast klasycznych zapytań http dobrym wyborem wydaje się być WebSocket. \cite{javascript}
\par Ostatnim elementem jest UI/UX. W założeniach projektowych wspomniano, że użytkownik powinien czuć się naturalnie korzystając z platformy. Gałąź UI/UX jest bardzo obszerna merytorycznie i wykracza poza ramy tej pracy, lecz by zachować jakąkolwiek estetykę - w pracy korzystano z biblioteki \textit{Material-ui}, która dostarcza predefiniowanych komponentów dla aplikacji webowych. Korzystając z tego nie należy przejmować się estetyką inputów, przycisków oraz wielu innych komponentów. \cite{materialui}
\section{Higiena projektu}
Kod powinien być napisany według aktualnych praktyk aby był on jak najbardziej czytelny. Wpływa to na mniejszą ilość błędów, łatwiejsze rozwijanie  oprogramowania oraz szybsze naprawy. Powinien być napisany w taki sposób by nowe osoby nie miały problemu ze zrozumieniem go. \\
Kolejnym aspektem jest modularność. Wynika ona z kultury pisania kodu, ale tutaj jest ona szczególnie ważna ponieważ zakłada się, że będzie się stale dodawać obsługę nowych urządzeń. Udostępniając API dla użytkowników, przekazuje się im narzędzia, które pozwalają przyłączać nowe urządzenia do platformy. Zachowanie dobrej jakości kodu pozwala tworzyć nowe moduły bez wiedzy o całym projekcie. Izolując logikę całego systemu sprawia się, że potencjalny użytkownik chcący stworzyć nowy moduł, powinien znać tylko kilka metod, które zostały mu opisane w API. 
\par Pomysł ten został zainspirowany narzędziami do tworzenia natywnych modułów w frameworku React Native, gdzie kod w Javie dla systemu Android można połączyć z javascriptowym frameworkiem. 
\section{Testy automatyczne}
Pisanie testów obejmujących cały kod projektu i wszystkie krytyczne ścieżki to bardzo pracochłonne przedsięwzięcie i wymagające pewnej wiedzy z zakresu \textit{Quality assurance}. Z tego powodu testy w tym projekcie zostały ograniczone tylko do pewnych krytycznych ścieżek działania aplikacji. By ograniczyć stopień skomplikowania, testy zostały również napisane w języku javascript z metodyką whitebox. Użyto takich popularnych biblioteki jak \textit{mocha} do uruchamiania testów i \textit{chai} do tworzenia ścieżek krytycznych i walidacji. \cite{chai, mocha}