\chapter{Analiza technologiczna}
Celem projektu było to aby jak największe grono odbiorców było w stanie tworzyć dodatkowe moduły do sterowania w związku z tym wykorzystano najpopularniejsze technologie. Należałoby również zadbać o jakość kodu i trzymanie się ustalonych zasad. Postępowanie takie ma długoterminowe korzyści w postaci mniejszej ilości błędów, łatwości ich naprawiania jak i szybsze wdrażanie nowych osób w ten projekt. 
\par Jak już zostało wspomniane w rozdziale 2, w trakcie pracy wzorowano się na istniejących platformach. Głównym elementem był UI i dominacja aplikacji webowych. Inaczej została zaprojektowana na przykład komunikacja która u innych jest oparta o REST API, a w projekcje zadecydowano na wykorzystanie websocket. Nie oznacza to że całkiem zrezygnowano z zapytań http, ponieważ na przykład konfiguracja websocket jest poprzedzana zapytaniem do serwera o port na którym ma działać.
\section{Zastosowane technologie}
Aktualnie jednym z najpopularniejszych języków jest JavaScript w nowych odsłonach ES6+. Jest to język skryptowy więc na różnych systemach i architekturach powinien działać tak samo. Największą wadą tego języka jest dynamiczne typowanie przez co kod potrafi być mało precyzyjny. W tym celu zdecydowałem się na nakładkę Typescript, która wprowadza interface'y znane m.in z języka Java i potrafi wymusić silne typowanie. Dodatkowo strona internetowa powinna być napisana w jednym z popularniejszych frameworków by oszczędzić czas i usprawnić prace nad projektem. Na rynku dominują teraz 3 biblioteki: React, Angular i Vue. Wybór padł na Reacta ze względu na duże wsparcie community i wsparcie składni JSX. \\
Jednym z ważniejszych aspektów platformy jest responsywność dlatego zamiast klasycznych zapytań http dobrym wyborem wydaje się być WebSocket. 
\par Ostatnim elementem jest UI/UX. W założeniach projektowych wspomniano że użytkownik powinien czuć się naturalnie korzystając z platformy. Gałąź UI/UX jest bardzo obszerna merytorycznie i wykracza poza ramy tej pracy, lecz by zachować jakąkolwiek estetykę to w pracy korzystano z biblioteki \textit{Material-ui} która dostarcza predefiniowanych komponentów dla aplikacji webowych. Korzystając z tego nie trzeba się przejmować estetyką inputów, przycisków i wielu innych innych komponentów. 
\section{Higiena projektu}
Kod powinien być napisany według aktualnych praktyk aby był on możliwie jak najbardziej czytelny. Wpływa to na zmniejszoną ilość błędów, łatwiejsze rozwijanie i szybsze naprawy. Powinien być napisany w taki sposób by nowe osoby nie miały problemu ze zrozumieniem. \\
Kolejnym aspektem jest modularność. Wynika ona z kultury pisania kodu, ale tutaj jest ona szczególnie ważna ponieważ zakładamy, że będziemy stale dodawać obsługę nowych urządzeń. Udostępniając API dla użytkowników, dajemy im narzędzia które pozwalają łączyć nowe urządzenia do platformy. Zachowanie dobrej jakości kodu pozwala tworzyć nowe moduły bez wiedzy o całym projekcie. Izolując logikę całego systemu sprawiamy że potencjalny użytkownik chcący stworzyć nowy moduł, powinien znać tylko kilka metod które mu opisaliśmy. 
\par Pomysł ten został zainspirowany narzędziami do tworzenia natywnych modułów w frameworku React Native, gdzie nasz kod w Javie dla systemu Android możemy połączyć z javascriptowym frameworkiem. 
\section{Testy automatyczne}
Pisanie testów obejmujących cały kod projektu i wszystkie krytyczne ścieżki to bardzo pracochłonne przedsięwzięcie i wymagające pewnej wiedzy z zakresu \textit{Quality assurance} dlatego testy w tym projekcie zostały ograniczone tylko to pewnych krytycznych ścieżek działania aplikacji. By ograniczyć stopień skomplikowania to Testy są również pisane w języku javascript z metodyką whitebox. Zostały użyte takie popularne biblioteki jak \textit{mocha} do uruchamiania testów i \textit{chai} do tworzenia ścieżek krytycznych i walidacji. 