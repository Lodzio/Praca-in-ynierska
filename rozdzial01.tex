\chapter{Wstęp}
\section{Cel pracy}
    Celem jest stworzenie platformy do automatyki która łączy urządzenia budynkowe.
\section{Zakres pracy}
Obiektem pracy jest oprogramowanie do wsparcia automatyki, lecz w celu przykładowej implementacji użyty został sterownik przekaźnikowy i urządzenie Raspberry.
\section{Wprowadzenie}
Platforma do automatyki domowej powinna być wysoko poziomowym projektem, który pozwala użytkownikom budynku na łatwy dostęp do urządzeń automatyki oraz do informacji płynących z czujników. Priorytetem tej pracy jest łatwość obsługi, elastyczność i responsywność. Jest to odpowiedź na rosnące zainteresowanie małymi i tanimi urządzeniami IoT oraz na potrzebę zintegrowania ich w jednym systemie. W dalszych rozdziałach mojej pracy wyjaśnię jakie technologie zostały wykorzystane oraz z jakiego powodu.  Przedstawię także jak rozwiązano problemy integracji urządzeń i opiszę przykładową implementację urządzenia.