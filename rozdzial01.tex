\chapter{Wstęp}
Automatyka budynkowa coraz bardziej przenika do naszego życia zawodowego oraz do naszych domów. Jej celem jest optymalizacja miejsca pracy, zmniejszenie poboru prądu, bezpieczeństwo oraz wygoda. Wraz z rozwojem technologii mamy możliwość coraz większej ingerencji w działanie budynku. Na rynku obserwujemy również spadek cen urządzeń automatyki. Od niedawna na rynku pojawiły się tanie moduły automatyki domowej, które cenowo można porównać do prostych urządzeń AGD. Motywacją do wykonania tego projektu było stworzenie oprogramowania, które łączy te moduły i sprawia, że klasa niższa i średnia społeczeństwa również mogłaby pozwolić sobie na atrybuty domu inteligentnego. \par
Platforma do automatyki domowej powinna być wysoko poziomowym projektem, który pozwala użytkownikom budynku na łatwy dostęp do urządzeń automatyki oraz do informacji płynących z czujników. Priorytetem tej pracy jest łatwość obsługi, elastyczność i responsywność. Jest to odpowiedź na rosnące zainteresowanie małymi i tanimi urządzeniami IoT oraz na potrzebę zintegrowania ich w jednym systemie. W dalszych rozdziałach pracy zostanie wyjaśnione jakie technologie zostały wykorzystane oraz z jakiego powodu.  Zostanie także przedstawione w jaki sposób rozwiązano problemy integracji urządzeń i oraz pojawi się opis przykładowej implementacji urządzenia.
\section{Cel pracy}
    Celem jest stworzenie platformy do automatyki, która łączy urządzenia budynkowe. Założenia projektu to elastyczność, łatwość konfiguracji i łatwy i szybki dostęp dla użytkowników budynku. Zostały one rozwinięte w dalszej części pracy. \newline
    Została również przeprowadzona analiza technologiczna, w której zostały uargumentowane decyzje o użytych technologiach. 
\section{Zakres pracy}
Obiektem pracy jest oprogramowanie do wsparcia automatyki, lecz w celu przykładowej implementacji użyty został sterownik przekaźnikowy, czujnik temperatury i urządzenie Raspberry. Przykładowa implementacja była obiektem testów i udowadniała działanie systemu w warunkach domowych. W ramach automatyzacji procesu rozwoju aplikacji, zostały stworzone testy automatyczne sprawdzające działanie poszczególnych komponentów. Ma to na celu wyeliminowanie błędów i pomoc przy rozwijaniu nowych funkcjonalności.
