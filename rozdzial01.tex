\chapter{Wstęp}
\section{Wprowadzenie}
Platforma do automatyki domowej powinna być wysoko poziomowym projektem który pomaga użytkownikom budynku w łatwy dostęp do urządzeń automatyki jak i informacji płynących z czujników. Priorytetem tej pracy jest łatwość obsługi, elastyczność i responsywność.Jest to odpowiedź na rosnące zainteresowanie małymi i tanimi urządzeniami IoT i na potrzebę zintegrowania ich w jednym systemie. W dalszych rozdziałach wyjaśnię jakie technologie zostały wykorzystane i dlaczego, jak rozwiązano problemy integracji urządzeń i opisana przykładowa implementacja urządzenia

% Projekt polega na stworzeniu oprogramowania do automatyki domowej, którą obsługujemy za pomocą strony internetowej. Za pomocą tej strony powinniśmy sterować urządzeniami w domu i obserwować dane z czujników. Z 
% Planuje się wzorować na rozwiązaniach open-source takich jak: Domoticz, Home Assistant, Node-red.