\chapter{Wstęp}
Automatyka budynkowa coraz bardziej przenika do naszego życia zawodowego czy w domu. Jej celem jest optymalizacja miejsca pracy, zmniejszenie poboru prądu, bezpieczeństwo lub wygoda. Wraz z rozwojem technologii mamy możliwość coraz większej ingerencji w działanie budynku, ale również obserwujemy taniejące ceny urządzeń automatyki. Od niedawna na rynku pojawiły się tanie moduły automatyki domowej które cenowo są porównywalne do prostych urządzeń AGD. Moją motywacją do stworzenia tego projektu jest stworzenie oprogramowania które łączy te urządzenia i sprawia że klasa niższa i średnia społeczeństwa również mogła sobie pozwolić na atrybuty domu inteligentnego.
Platforma do automatyki domowej powinna być wysoko poziomowym projektem, który pozwala użytkownikom budynku na łatwy dostęp do urządzeń automatyki oraz do informacji płynących z czujników. Priorytetem tej pracy jest łatwość obsługi, elastyczność i responsywność. Jest to odpowiedź na rosnące zainteresowanie małymi i tanimi urządzeniami IoT oraz na potrzebę zintegrowania ich w jednym systemie. W dalszych rozdziałach mojej pracy wyjaśnię jakie technologie zostały wykorzystane oraz z jakiego powodu.  Przedstawię także jak rozwiązano problemy integracji urządzeń i opiszę przykładową implementację urządzenia.
\section{Cel pracy}
    Celem jest stworzenie platformy do automatyki która łączy urządzenia budynkowe. Założenia projektu to elastyczność, łatwość konfiguracji i łatwy i szybki dostęp dla użytkowników budynku. Zostały one rozwinięte w dalszej części pracy. \newline
    Została również przeprowadzona analiza technologiczna w której zostały uargumentowane decyzje o użytych technologiach. 
\section{Zakres pracy}
Obiektem pracy jest oprogramowanie do wsparcia automatyki, lecz w celu przykładowej implementacji użyty został sterownik przekaźnikowy, czujnik temperatury i urządzenie Raspberry. Przykładowa implementacja była obiektem testów i udowadnia działanie systemu w warunkach domowych. W ramach automatyzacji procesu rozwoju aplikacji, zostały stworzone testy automatyczne sprawdzające działanie poszczególnych komponentów. Ma to na celu wyeliminowanie błędów i pomoc przy rozwijaniu nowych funkcjonalności.
