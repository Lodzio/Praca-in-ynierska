\chapter{Podsumowanie}
\label{chap:podsumowanie}
Celem pracy było stworzenie oprogramowania wspierającego automatykę budynkową. Założeniem tego projektu była elastyczność pod względem łączonych urządzeń i pod względem systemu na jakim działa. 
Łatwa konfiguracja i implementacja nowych urządzeń.
Łatwy i szybki dostęp do sterowania dla użytkowników budynku.
Założenia zostały spełnione, co zostało pokazane w przykładowej implementacji.
\newline
Stworzona platforma jest jeszcze pełna niedoskonałości i na rynku istnieją lepsze systemy, lecz podstawowe cele zostały spełnione i technicznie jest to oprogramowanie gotowe do implementacji i użytku. Zostało to udowodnione na przykładowej implementacji, która pokazuje wartość tego oprogramowania. Program został stworzony przy pomocy nowoczesnych technologii i praktyk. W celu dalszego rozwoju tego produktu warto rozważyć następujące sprawy:
\begin{itemize}
    \item poprawa UI/UX,
    \item refactoring kodu,
    \item wsparcie większej ilości urządzeń,
    \item wsparcie większej ilości inteface'ów,
    \item stworzenie narzędzi do łatwiejszej implementacji dla ludzi bez wiedzy programistycznej (na przykład dla elektryków),
    \item zabezpieczenie panelu przed niepowołanymi osobami,
    \item poprawa aktualnych błędów i dodawanie dodatkowych funkcjonalności dla poprawy wygody użytkowników.
\end{itemize}
Jako, że istnieje kilka rozbudowanych i darmowych platform to sens ekonomiczny tego projektu jest znikomy. Warto rozważyć hostowanie platformy dla klientów, aby nie musieli oni inwestować w domowy serwer. Można by wziąść pod uwagę małą jednorazową opłatę dla użytku własnego, lecz wtedy powinno ograniczyć się ingerencję w kod. Na podstawie doświadczeń warto rozważyć usunięcie nakładki typescript z aplikacji webowej, a zaimplementować go na serwerze. 