\chapter{Wymagania systemu}
Platforma do automatyki powinna być elastycznym tworem by można było ją implementować w różnych konfiguracjach. niezależnie od ilości i typów urządzeń automatyki. Powinna tez działać na różnych systemach by można było postawić na serwerze wirtualnym, lokalnym PC czy urządzeniu embedded. Drugą najważniejszą cechą powinna być łatwość konfiguracji by przy jak najmniejszej wiedzy zintegrować istniejących urządzenia do automatyki. Algorytm powinien być jak najprostszy ograniczający się do wybrania modelu urządzenia lub sprawdzenia interface'u po jakim się komunikuje i wprowadzenia go do systemu. Kolejną najważniejsza rzeczą powinna być łatwość dostępu dla użytkowników by czerpali oni maksymalne korzyści z inteligencji budynku, a jednocześnie nie odczuwali negatywnych aspektów takich jak na przykład włożony czas w wykonywanie podstawowych czynności czy wymagane skupienie intelektualne w użytkowanie. System powinien być dla nas naturalny tak jak każdy klasyczny włącznik w domu. Przy dzisiejszym przywiązaniu do telefonów komórkowych wydaje się to idealnym środowiskiem do kontrolowania budynkiem przez np aplikacje webową, mobilną czy jakiś widget który byłby tak naturalny jak zegarek w telefonie. 