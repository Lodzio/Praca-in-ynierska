\chapter{Wymagania systemu}
Należy podkreślić, że obiekt pracy, którym jest platforma do automatyki to tylko oprogramowanie. Z tego wynika najważniejsza cecha czyli, że platforma powinna być elastycznym tworem aby można było ją implementować w różnych konfiguracjach,  niezależnie od ilości i typów urządzeń automatyki. Powinna również działać na różnych systemach, aby można było postawić ją na serwerze wirtualnym, lokalnym PC czy urządzeniu embedded. Ma to bezpośredni wpływ na liczbę potencjalnych odbiorców. \par Drugą najważniejszą cechą powinna być łatwość konfiguracji, aby przy jak najmniejszej wiedzy zintegrować istniejące urządzenia automatyki. Algorytm powinien być jak najprostszy, ograniczający się do wybrania modelu urządzenia lub sprawdzenia interface'u po jakim się komunikuje i wprowadzenia go do systemu. 
\par Kolejną najważniejsza rzeczą powinna być łatwość dostępu dla użytkowników by czerpali oni maksymalne korzyści z inteligencji budynku, a jednocześnie nie odczuwali negatywnych aspektów takich jak włożony czas w wykonywanie podstawowych czynności, czy wymagane skupienie intelektualne w użytkowaniu. System powinien być tak naturalny, jak każdy klasyczny włącznik w domu. Przy dzisiejszym przywiązaniu do telefonów komórkowych wydaje się to idealnym środowiskiem do kontrolowania budynku np. poprzez aplikacje webową, mobilną czy widget, który mógłby tak wniknąć do naszej codzienności jak zegarek w telefonie. \par
Wpasowanie się w dzisiejsze trendy uatrakcyjnia produkt dla przyszłych klientów jak i developerów, a te trendy pokazują, że środowisko aplikacji webowych wydaje się dobrym wyborem. Dzięki wykorzystaniu nowoczesnych i popularnych technologii projekt ma szansę na długoterminowy rozwój. Objawia się to obszerną społecznością, która wzbogaca internet w ilość poruszanych problemów i ich rozwiązania. Praca w popularnym środowisku gwarantuje również dużą ilość potencjalnych programistów posiadających wymaganą wiedzę do rozwoju projektu.
\par Tworzenie nowoczesnego projektu łączy się z kolejnym ważnym aspektem czyli testami. Projekt powinien je zawierać by zautomatyzować i przyspieszyć wprowadzanie nowych rozwiązań oraz wyszukiwanie krytycznych błędów. 
\par Podsumowując, tworząc ten projekt skupiono się nie tylko na funkcjonalności platformy, lecz również na długoterminowym rozwoju. Projekt powinien być elastyczny, łatwy w konfiguracji i zapewniać szybką i łatwą kontrolę nad budynkiem. 