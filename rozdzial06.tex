\chapter{Przykłady działania}
W trakcie prac został podłączony po mqtt sterownik przekaźnikowy, który sterował lampką oraz czujnik temperatury DS18B20 do platformy, która działała na urządzeniu Raspberry Pi oraz na laptopie. Płytka Raspberry została wykorzystana z kilku względów. 
\begin{itemize}
    \item Małe rozmiary dzięki którym łatwiej nam znaleźć dedykowane stałe miejsce.
    \item Małe zapotrzebowanie na energie, które ma bardzo duży wpływ w przypadku gdy urządzenie działa 24 godziny na dobę.
    \item Mała cena, która wynosi ok 200 zł za wersje 3B+.
\end{itemize}
Oczywiście to urządzenie ma dwie spore wady.
\begin{itemize}
    \item Mała stabilność, która jest efektem tego, że system działa na karcie SD.
    \item Do tego urządzenia musimy sami dostarczyć zasilanie 5V o minimalnym natężeniu 2A. Często z winy słabej jakości kabla albo zasilacza awaryjność powoduje, że musimy restartować urządzenie raz na miesiąc czy nawet na tydzień.
\end{itemize}
Wady i zalety pokazują, że jest to świetna konfiguracja prototypownia ,lecz nie do długoterminowej instalacji.